% ****** Start of file apssamp.tex ******
%
%   This file is part of the APS files in the REVTeX 4.2 distribution.
%   Version 4.2a of REVTeX, December 2014
%
%   Copyright (c) 2014 The American Physical Society.
%
%   See the REVTeX 4 README file for restrictions and more information.
%
% TeX'ing this file requires that you have AMS-LaTeX 2.0 installed
% as well as the rest of the prerequisites for REVTeX 4.2
%
% See the REVTeX 4 README file
% It also requires running BibTeX. The commands are as follows:
%
%  1)  latex apssamp.tex
%  2)  bibtex apssamp
%  3)  latex apssamp.tex
%  4)  latex apssamp.tex
%
\documentclass[%
 reprint,
%superscriptaddress,
%groupedaddress,
%unsortedaddress,
%runinaddress,
%frontmatterverbose, 
%preprint,
%preprintnumbers,
%nofootinbib,
%nobibnotes,
%bibnotes,
 amsmath,amssymb,
 aps,
%pra,
%prb,
%rmp,
%prstab,
%prstper,
%floatfix,
]{revtex4-2}

\usepackage{graphicx}% Include figure files
\usepackage{dcolumn}% Align table columns on decimal point
\usepackage{bm}% bold math
\usepackage[english]{babel}
\makeatletter
\@namedef{l@en}{\l@english}
\makeatother
%\usepackage{hyperref}% add hypertext capabilities
%\usepackage[mathlines]{lineno}% Enable numbering of text and display math
%\linenumbers\relax % Commence numbering lines

%\usepackage[showframe,%Uncomment any one of the following lines to test 
%%scale=0.7, marginratio={1:1, 2:3}, ignoreall,% default settings
%%text={7in,10in},centering,
%%margin=1.5in,
%%total={6.5in,8.75in}, top=1.2in, left=0.9in, includefoot,
%%height=10in,a5paper,hmargin={3cm,0.8in},
%]{geometry}

\newcommand{\pcc}{\,cm$^{-3}$}	% per cm-cubed
\newcommand{\ga}{$\text{Ga}_2\text{O}_3$}

\begin{document}

\preprint{APS/123-QED}

\title{Manuscript Title:\\with Forced Linebreak}% Force line breaks with \\
\thanks{A footnote to the article title}%

\author{Ann Author}
 \altaffiliation[Also at ]{Physics Department, XYZ University.}%Lines break automatically or can be forced with \\
\author{Second Author}%
 \email{Second.Author@institution.edu}
\affiliation{%
 Authors' institution and/or address\\
 This line break forced with \textbackslash\textbackslash
}%

\collaboration{MUSO Collaboration}%\noaffiliation

\author{Charlie Author}
 \homepage{http://www.Second.institution.edu/~Charlie.Author}
\affiliation{
 Second institution and/or address\\
 This line break forced% with \\
}%
\affiliation{
 Third institution, the second for Charlie Author
}%
\author{Delta Author}
\affiliation{%
 Authors' institution and/or address\\
 This line break forced with \textbackslash\textbackslash
}%

\collaboration{CLEO Collaboration}%\noaffiliation

\date{\today}% It is always \today, today,
             %  but any date may be explicitly specified

\begin{abstract}
An article usually includes an abstract, a concise summary of the work
covered at length in the main body of the article. 
\begin{description}
\item[Usage]
Secondary publications and information retrieval purposes.
\item[Structure]
You may use the \texttt{description} environment to structure your abstract;
use the optional argument of the \verb+\item+ command to give the category of each item. 
\end{description}
\end{abstract}

%\keywords{Suggested keywords}%Use showkeys class option if keyword
                              %display desired
\maketitle

%\tableofcontents

\section{Introduction}

Gallium oxide (\ga), often regarded as a representative of the fourth generation of 
semiconductors, has garnered widespread interest due to its unique physical and 
electronic properties.
With an ultra-wide bandgap, high breakdown electric field, and transparency in the 
deep ultraviolet region, \ga\ holds significant promise for applications in high-power 
electronics, UV photodetectors, and transparent conducting oxides. 
In addition to its thermodynamically stable monoclinic $\beta$ phase, \ga\ exhibits a 
rich polymorphism, including metastable $\alpha$ (corundum-type), $\gamma$ 
(defective spinel), $\delta$ (bixbyite), and $\kappa$ (orthorhombic) phases. 
This polymorphic diversity opens new avenues for directional material design and property 
tuning but also presents formidable challenges in controlled synthesis, phase stability, 
and interface engineering.
Consequently, a comprehensive theoretical understanding of \ga---particularly of its 
complex polymorphic nature---is essential for realizing its full potential in practical 
applications.

Early theoretical investigations of \ga\ predominantly focused on its electronic structure 
using first-principles methods \cite{he_first-principles_2006,xu_structure_2007}, 
laying the groundwork for understanding its doping behavior \cite{varley_oxygen_2010}, 
lattice dynamics \cite{yan_phonon_2018,yang_lattice_2024}, 
interfaces \cite{bermudez_structure_2006}, and phase stability \cite{fan_low-energy_2022}. 
While first-principles approaches continue to offer critical insights, their high computational 
cost and limited scalability pose significant challenges for modeling large-scale phenomena, 
such as irradiation-induced phase transitions or extended defect evolution 
\cite{azarov_disorder-induced_2021,han_unraveling_2025}. 
Conversely, molecular dynamics (MD) enables simulations over larger time 
and length scale; however, the empirical interatomic potentials 
it relies on often struggle to accurately capture the complex potential energy landscape of 
\ga's polymorphic phases \cite{blanco_energetics_2005}. 

In recent years, the rapid advancement of artificial intelligence has ushered in a 
paradigm shift in materials science \cite{pyzer-knapp_accelerating_2022}. 
Among the most transformative developments is the emergence of machine learning interatomic 
potentials (MLIPs), which offer near-DFT accuracy combined with excellent scalability, 
making them powerful alternatives to traditional empirical potentials 
in MD simulations \cite{ko_recent_2023}. 
Motivated by these advantages, several pioneering research groups have made 
continuous efforts to develop MLIPs tailored for \ga, spanning a wide range of polymorphic 
phases---from the thermodynamically stable $\beta$ phase \cite{li_deep_2020,liu_machine_2020,zhao_phase_2021} 
to metastable $\alpha$ \cite{sun_neuroevolution_2023}, $\kappa$ \cite{wang_dissimilar_2024}, 
$\epsilon$ \cite{sun_neuroevolution_2023}, and amorphous configurations \cite{liu_unraveling_2023}. 
A landmark achievement was realized by Zhao et al. \cite{zhao_complex_2023}, 
who introduced a general-purpose MLIP referred to as tabGAP, capable of accurately 
describing the atomic interactions across all major \ga\ polymorphs. 
This potential has enabled large-scale simulations of 
defect evolution \cite{he_threshold_2024,he_ultrahigh_2024,liu_orientation-dependent_2025}, 
phase transformations \cite{azarov_self-assembling_2025}, 
and crystallization dynamics \cite{zhang_orientation-dependent_2023,li_edge-dependent_2025}, 
and has played a pivotal role in recent studies of irradiation effects in 
\ga\ \cite{azarov_universal_2023,zhao_crystallization_2025,han_electronic_2025,abdullaev_ions_2025}.
However, despite its considerable success, tabGAP still exhibits notable limitations compared 
to other state-of-the-art MLIPs, including relatively lower accuracy and computational 
speed, as will be demonstrated in the following sections. 
These limitations motivate us to rigorously benchmark the overall 
performance of tabGAP and develop a new, more accurate and efficient MLIP for \ga.

\section{Methodology}

\subsection{NEP architecture}

NEP, developed by Fan \textit{et al.}, consists of an ACE-like descriptor constructed using 
Chebyshev polynomials and a regression model based on 
a feedforward neural network \cite{fan_neuroevolution_2021,fan_gpumd_2022,song_general-purpose_2024}. 
Its explicit atomic-environment featurization is carefully designed to strike an excellent 
balance between prediction accuracy and computational efficiency. 
Moreover, NEP is natively implemented in the Graphics Processing Units Molecular Dynamics 
(GPUMD) package, enabling an impressive computational throughput of up to $1\times10^{7}$ 
atom-steps per second on a single GPU \cite{xu_gpumd_2025}. 
Considering its overall performance, NEP stands out as one of the most promising 
MLIPs for large-scale simulations of \ga\ under a wide range of physical conditions.

The radial function $g_n(r_{ij})$, which serves as the basic building block of the NEP
descriptor, is defined as
\begin{equation}
g_n(r_{ij}) = \sum_{k} c^{ij}_{nk} f_k(r_{ij}),
\end{equation}
where $r_{ij}$ is the distance between atoms $i$ and $j$, 
$c^{ij}_{nk}$ is a trainable parameter, and $f_k(r_{ij})$ is a Chebyshev-based
radial basis function with a finite cutoff:
\begin{equation}
f_k(r_{ij}) = \frac{1}{2}\!\left[T_k\!\!\left(2(r_{ij}/r_{\mathrm{cut}} - 1)^2 - 1\right)+1\right] s_c(r_{ij}),
\end{equation}
where $T_k(x)$ is the $k$-th-order Chebyshev polynomial of the first kind, and 
$s_c(r_{ij})$ is a cutoff function
\begin{equation}
s_c(r_{ij}) =
\begin{cases}
  \frac{1}{2}\!\left[1+\cos(\pi r_{ij}/r_{\mathrm{cut}})\right], & r_{ij} \le r_{\mathrm{cut}}, \\
  0, & r_{ij} > r_{\mathrm{cut}}.
\end{cases}
\end{equation}

The $n$-th radial descriptor of atom $i$ is then constructed as
\begin{equation}
q_n^i = \sum_{j \ne i} g_n(r_{ij}),
\end{equation}
and the $n$-th angular descriptor of atom $i$ with angular-momentum index $l$
is defined as
\begin{equation}
q_{n,l}^i = \sum_{j \ne i}\sum_{k \ne i} g_n(r_{ij})\,g_n(r_{ik})\,P_l(\theta_{ijk}),
\end{equation}
where $P_l(\theta_{ijk})$ is the $l$-th-order Legendre polynomial of the angle
$\theta_{ijk}$ between atoms $i$, $j$, and $k$.

All descriptors are concatenated into a single vector, which is then passed
through a feedforward neural network to predict the atomic potential energy:
\begin{equation}
U^i =
\sum_{\mu=1}^{N_{\mathrm{neu}}}
\omega^{(1)}_\mu
\tanh\!\left(
\sum_{\nu=1}^{N_{\mathrm{des}}}
\omega^{(0)}_{\mu\nu} q^i_\nu - b^{(0)}_\mu
\right)
- b^{(1)},
\end{equation}
where $N_{\mathrm{neu}}$ is the number of neurons in the hidden layer,
$N_{\mathrm{des}}$ is the number of descriptors,
$\omega^{(1)}_\mu$ and $b^{(1)}$ are the output-layer weight and bias,
$\omega^{(0)}_{\mu\nu}$ and $b^{(0)}_\mu$ are the hidden-layer weights and biases,
and $q^i_\nu$ denotes the $\nu$-th descriptor of atom $i$.

\subsection{Hyperparameters}

In this work, we use the fourth generation of NEP \cite{song_general-purpose_2024} to construct 
the MLIP for \ga. To faithfully capture the complex potential energy landscape of \ga, 
we adopt a relatively large parameterization strategy. 
Specifically, 9 radial descriptors ($r_\text{cut}=6\,\text{\AA}$) and 
42 angular descriptors ($r_\text{cut}=4\,\text{\AA}$) are employed, 
with a single hidden layer containing 100 neurons. 
The maximum Chebyshev polynomial orders are set to 12 and 10 for the radial and angular 
descriptors, respectively. 
In addition, 4-body and 5-body angular terms are included as described in Ref.~\cite{fan_gpumd_2022}. 
In total, the resulting model contains 11,377 trainable parameters. 
We also tested larger parameter settings; however, they did not provide noticeable improvements 
in accuracy while causing a substantial increase in computational cost.

A robust training-weighting scheme is employed to balance the accuracy across distinct atomic 
environments, which is essential for achieving successful training over an extremely wide range 
of average atomic energies, spanning from the most stable $\beta$ phase to highly unstable, nonstoichiometric 
amorphous states under high pressure. 
Our weighting strategy consists of two components.
First, we adopt the phase-dependent scaling scheme proposed in Ref.~\cite{zhao_complex_2023}, 
which enhances the descriptive capability of the model for key \ga\ phases, including its 
metastable crystalline structures.
Second, we further rescale the training weights of all configurations according to their 
average potential energy per atom, ensuring that the NEP model primarily focuses on low-energy 
configurations while still retaining reasonable descriptive capability for high-energy states.
The rescaling factor $\mathcal{F}$ is defined as
\begin{equation}
\label{eq:rescaling}
\mathcal{F} = \frac{1}{\left|e_i/\epsilon - e_{\beta\text{-phase}}/\epsilon\right|^{\alpha} + 1},
\end{equation}
where $e_i$ is the average potential energy per atom of the $i$-th configuration, 
$e_{\beta\text{-phase}}$ is that of the $\beta$ phase, 
and $\alpha$ and $\epsilon$ are small constants controlling the strength of the rescaling.
The average energy, atomic forces, and virial of each training configuration are used as training targets, 
with global weights of $\lambda_e=\lambda_f=\lambda_v=1$, following Ref.~\cite{liang_nep89_2025}. 
Both $\mathcal{L}_1$ and $\mathcal{L}_2$ regularizations are applied with $\lambda_1=\lambda_2=0.05$ to 
avoid overfitting \cite{fan_gpumd_2022}.

\subsection{Datasets}

The dataset reported in Ref.~\cite{zhao_complex_2023}, which was originally used to construct 
soapGAP and tabGAP models for \ga\ and is hereafter referred to as the GAP dataset, 
serves as the main part of our training set.
The GAP dataset contains 1,630 configurations covering a wide range of Ga--O systems, 
from crystalline \ga\ to nonstoichiometric, amorphous, and molten states, 
and also includes a number of few-atom systems for calibration.
This dataset provides a rich collection of representative atomic configurations, although with 
a relatively limited number of samples. 
It is generally sufficient for broad, general-purpose applications where ultimate accuracy 
is not the primary concern.

However, we find that the GAP dataset still lacks certain key configurations required 
to recover essential physical features in our irradiation simulations. 
To address this issue, we augment the dataset with additional training configurations generated 
using exactly the same sampling protocol as for the GAP dataset.
These newly sampled configurations are found to lie beyond the interpolation capability of tabGAP 
and are crucial for accurately capturing SHI irradiation-induced phenomena.
Two types of configurations are added. 
The first type comprises structures along the energy--volume 
equation-of-state curve of the $\gamma$ phase. 
This addition is motivated by the fact that 
the $\beta \to \gamma$ transition is widely observed in radiation experiments on \ga\ 
\cite{zhao_crystallization_2025,han_electronic_2025,abdullaev_ions_2025}, 
yet $\gamma$-phase configurations are underrepresented in the original GAP dataset. 
The second type consists of configurations sampled from heating--cooling processes 
of the $\beta$ phase under various pressure and volume conditions,
because such thermodynamic pathways frequently occur during irradiation.
Our principle component analysis (PCA) shows that these newly added configurations 
effectively fill the gap in the configuration space between the $\beta$ phase 
and the molten/disordered states that exists in the original GAP dataset.

\subsection{Simulations}

The SHI irradiation simulations are performed using the GPUMD package~\cite{xu_gpumd_2025},
following the strategy proposed in Ref.~\cite{rymzhanov_damage_2017}.
In general, the simulations proceed as follows.
(1) The energy deposition from SHIs in \ga\ is simulated using TREKIS, a Monte Carlo code
developed to model the electronic kinetics following SHI impact on matter
\cite{medvedev_time-resolved_2015,rymzhanov_effects_2016}.
(2) The resulting radial energy profile of the ion track is then used to initialize the atomic
velocities of the \ga\ lattice for the subsequent MD simulations.
(3) The MD simulations are carried out in the NVE ensemble, while the boundary temperature
is controlled using Nose-Hoover chain thermostats, with the coupling parameters calibrated
as shown in Fig.~S1.
Further methodological details regarding TREKIS are provided in Appendix~\ref{app:trekis}.


\begin{acknowledgments}
We wish to acknowledge the support of the author community in using
REV\TeX{}, offering suggestions and encouragement, testing new versions,
\dots.
\end{acknowledgments}

\appendix

\section{Methodology of TREKIS}
\label{app:trekis}
The central parameter in TREKIS is the differential scattering cross-section 
$\sigma$, which describes the interactions between ballistic electrons and 
orbital electrons. 
This parameter is calculated using the DSF-CDF formalism, based 
on optical data obtained from photo-absorption experiments: 
\begin{equation} 
\frac{\partial^{2}\sigma}{\partial(\hbar\omega)\partial(\hbar q)} = 
\frac{2[Z_{\text{e}}(v,q)e]^{2}}{\pi\hbar^{2}v^{2}} \frac{1}{\hbar 
q}\mathrm{Im}[\frac{-1}{\epsilon(\omega,q)}]\ , \label{eq1} 
\end{equation} 
In this formalism, $\hbar\omega$ represents the transferred energy, $q$ denotes 
the transferred momentum, $v$ is the ion velocity, $Z_e$ is the effective 
charge of the ion passing through \ga, and $\epsilon$ is the complex 
dielectric function. 
The ion charge $Z_{\text{e}}(v,q)$ is calculated with Barkas formula 
\cite{Gervais1994}: 
\begin{equation} 
Z_{\text{e}}(v,q) = Z_{\text{ion}}(1-\exp(-\frac{v}{v_{0}}Z_{\text{ion}}^{-\frac{2}{3}})) 
\end{equation} 
where $v$ is the velocity of the ion, $v_{0}=c/125$ is the empirically atomic 
electron velocity and $Z_{\text{ion}}$ is the ion charge in full ionization. 
The optical energy loss function of \ga\ is taken from 
Ref.~\cite{he_first-principles_2006}, and fitted with a finite sum of Drude-Lorentz 
oscillator functions: 
\begin{equation} 
\mathrm{Im}(\frac{-1}{\epsilon(\omega,0)}) = \sum_{i} 
\frac{A_{\mathrm{i}}\gamma_{i}\hbar\omega}{(\hbar^{2}\omega^{2}-E_{i}^{2})^{2}+(\gamma_{i}\hbar\omega)^{2}} 
\end{equation} 
where $A_{i}$, $\gamma_{i}$, and $E_{i}$ are the fitted parameters, which are 
constrained by two sum rules: 
\begin{enumerate} 
\item The f-sum rule: 
\begin{equation} 
\frac{2}{\pi 
\Omega_{\text{p}}^{2}}\int_{I_{\text{p}}}^{\infty}\mathrm{Im}(\frac{-1}{\epsilon(\omega,0)})_{\text{shell}}\omega 
\mathrm{d}\omega = N_{\text{e,shell}} 
\end{equation} 
where $\Omega_{\text{p}}^{2}=4\pi e^{2}n_{at}/m_{\text{e}}$ is the plasma frequency, 
$N_{\text{e,shell}}$ is the number of electrons in selected shell, and $I_{\text{p}}$ is 
ionization potential. 
\item The KK-sum rule: 
\begin{equation} 
\frac{2}{\pi}\int_{0}^{\infty} 
\mathrm{Im}(\frac{-1}{\epsilon(\omega,0)})\frac{d\omega}{\omega} = 1 
\end{equation} 
\end{enumerate} 
The free-electron approximation is used to calculated the dispersion relation 
for the oscillator energy $E_{i}$ and transferred momentum q: 
\begin{equation} 
E_{i}(q) = E_{i}(0) + \frac{\hbar^{2}q^{2}}{2m_{\text{e}}} 
\end{equation} 
The lower and upper limits of the transferred energy during an inelastic 
scattering event are as follows: 
\begin{equation} 
W_{-} = I_{\text{p}} 
\end{equation} 
\begin{equation} 
W_{+} = \frac{4Em_{1}m_{2}}{(m_{1}+m_{2})^{2}} 
\end{equation} 
where $E$ is the incident energy, and $m_{1}$ and $m_{2}$ are the masses of the 
two scattering particles. 
For the elastic scattering between electrons and atoms, i.e., the 
electron-phonon interaction, the Mott scattering cross section is employed with 
a modified Moli\`ere screening parameter $K_{\text{scr}}$ \cite{Jenkins2012}, given 
by the following equation: 
\begin{equation} 
\sigma_{\text{elastic}} = \sigma_{\text{Mott}}K_{\text{scr}} 
\end{equation} 
The upper limit of transferred energy in the elastic scattering event is: 
\begin{equation} 
W_{+} = \mathrm{min}(\frac{4Em_{1}m_{2}}{(m_{1}+m_{2})^{2}}, 
\hbar\omega_{\text{Debye}}) 
\end{equation} 
where $\omega_{\text{Debye}}=(6\pi^{2}n_{at})^{\frac{1}{3}}v_{s}$ is the Debye 
frequency and $v_{s}$ is the speed of sound in \ga. 
 
Several additional parameters were incorporated into the TREKIS calculations: 
the speed of sound was set to $5,250\,\mathrm{m/s}$ \cite{wright_acoustic_2025}, the density 
of \ga\ to $5.88\,\mathrm{g/cm}^3$, and the band gap to 
$4.9\,\mathrm{eV}$ \cite{zhang_tight-binding_2022}.

% The \nocite command causes all entries in a bibliography to be printed out
% whether or not they are actually referenced in the text. This is appropriate
% for the sample file to show the different styles of references, but authors
% most likely will not want to use it.
% \nocite{*}

\bibliography{apssamp}% Produces the bibliography via BibTeX.

\end{document}
%
% ****** End of file apssamp.tex ******
