% ****** Start of file apssamp.tex ******
%
%   This file is part of the APS files in the REVTeX 4.2 distribution.
%   Version 4.2a of REVTeX, December 2014
%
%   Copyright (c) 2014 The American Physical Society.
%
%   See the REVTeX 4 README file for restrictions and more information.
%
% TeX'ing this file requires that you have AMS-LaTeX 2.0 installed
% as well as the rest of the prerequisites for REVTeX 4.2
%
% See the REVTeX 4 README file
% It also requires running BibTeX. The commands are as follows:
%
%  1)  latex apssamp.tex
%  2)  bibtex apssamp
%  3)  latex apssamp.tex
%  4)  latex apssamp.tex
%
\documentclass[%
 reprint,
%superscriptaddress,
%groupedaddress,
%unsortedaddress,
%runinaddress,
%frontmatterverbose, 
%preprint,
%preprintnumbers,
%nofootinbib,
%nobibnotes,
%bibnotes,
 amsmath,amssymb,
 aps,
%pra,
%prb,
%rmp,
%prstab,
%prstper,
%floatfix,
]{revtex4-2}

\usepackage{graphicx}% Include figure files
\usepackage{dcolumn}% Align table columns on decimal point
\usepackage{bm}% bold math
\usepackage[english]{babel}
\makeatletter
\@namedef{l@en}{\l@english}
\makeatother
%\usepackage{hyperref}% add hypertext capabilities
%\usepackage[mathlines]{lineno}% Enable numbering of text and display math
%\linenumbers\relax % Commence numbering lines

%\usepackage[showframe,%Uncomment any one of the following lines to test 
%%scale=0.7, marginratio={1:1, 2:3}, ignoreall,% default settings
%%text={7in,10in},centering,
%%margin=1.5in,
%%total={6.5in,8.75in}, top=1.2in, left=0.9in, includefoot,
%%height=10in,a5paper,hmargin={3cm,0.8in},
%]{geometry}

\newcommand{\pcc}{\,cm$^{-3}$}	% per cm-cubed
\newcommand{\ga}{$\text{Ga}_2\text{O}_3$}

\begin{document}

\preprint{APS/123-QED}

\title{Manuscript Title:\\with Forced Linebreak}% Force line breaks with \\
\thanks{A footnote to the article title}%

\author{Ann Author}
 \altaffiliation[Also at ]{Physics Department, XYZ University.}%Lines break automatically or can be forced with \\
\author{Second Author}%
 \email{Second.Author@institution.edu}
\affiliation{%
 Authors' institution and/or address\\
 This line break forced with \textbackslash\textbackslash
}%

\collaboration{MUSO Collaboration}%\noaffiliation

\author{Charlie Author}
 \homepage{http://www.Second.institution.edu/~Charlie.Author}
\affiliation{
 Second institution and/or address\\
 This line break forced% with \\
}%
\affiliation{
 Third institution, the second for Charlie Author
}%
\author{Delta Author}
\affiliation{%
 Authors' institution and/or address\\
 This line break forced with \textbackslash\textbackslash
}%

\collaboration{CLEO Collaboration}%\noaffiliation

\date{\today}% It is always \today, today,
             %  but any date may be explicitly specified

\begin{abstract}
An article usually includes an abstract, a concise summary of the work
covered at length in the main body of the article. 
\begin{description}
\item[Usage]
Secondary publications and information retrieval purposes.
\item[Structure]
You may use the \texttt{description} environment to structure your abstract;
use the optional argument of the \verb+\item+ command to give the category of each item. 
\end{description}
\end{abstract}

%\keywords{Suggested keywords}%Use showkeys class option if keyword
                              %display desired
\maketitle

%\tableofcontents

\section{Introduction}

Gallium oxide (\ga), often regarded as a representative of the fourth generation of 
semiconductors, has garnered widespread interest due to its unique physical and 
electronic properties.
With an ultra-wide bandgap, high breakdown electric field, and transparency in the 
deep ultraviolet region, \ga\ holds significant promise for applications in high-power 
electronics, UV photodetectors, and transparent conducting oxides. 
In addition to its thermodynamically stable monoclinic $\beta$-phase, \ga\ exhibits a 
rich polymorphism, including metastable $\alpha$ (corundum-type), $\gamma$ 
(defective spinel), $\delta$ (bixbyite), and $\kappa$ (orthorhombic) phases. 
This polymorphic diversity opens new avenues for directional material design and property 
tuning but also presents formidable challenges in controlled synthesis, phase stability, 
and interface engineering.
Consequently, a comprehensive theoretical understanding of \ga---particularly of its 
complex polymorphic nature---is essential for realizing its full potential in practical 
applications.

Early theoretical investigations of \ga\ predominantly focused on its electronic structure 
using first-principles methods \cite{he_first-principles_2006,xu_structure_2007}, 
laying the groundwork for understanding its doping behavior \cite{varley_oxygen_2010}, 
lattice dynamics \cite{yan_phonon_2018,yang_lattice_2024}, 
interfaces \cite{bermudez_structure_2006}, and phase stability \cite{fan_low-energy_2022}. 
While first-principles approaches continue to offer critical insights, their high computational 
cost and limited scalability pose significant challenges for modeling large-scale phenomena, 
such as irradiation-induced phase transitions or extended defect evolution 
\cite{azarov_disorder-induced_2021,han_unraveling_2025}. 
Conversely, molecular dynamics (MD) enables simulations over larger time 
and length scale; however, the empirical interatomic potentials 
it relies on often struggle to accurately capture the complex potential energy landscape of 
\ga's polymorphic phases \cite{blanco_energetics_2005}. 

In recent years, the rapid advancement of artificial intelligence has ushered in a 
paradigm shift in materials science \cite{pyzer-knapp_accelerating_2022}. 
Among the most transformative developments is the emergence of machine learning interatomic 
potentials (MLIPs), which offer near-DFT accuracy combined with excellent scalability, 
making them powerful alternatives to traditional empirical potentials 
in MD simulations \cite{ko_recent_2023}. 
Motivated by these advantages, several pioneering research groups have made 
continuous efforts to develop MLIPs tailored for \ga, spanning a wide range of polymorphic 
phases---from the thermodynamically stable $\beta$-phase \cite{li_deep_2020,liu_machine_2020} 
to metastable $\alpha$ \cite{sun_neuroevolution_2023}, $\kappa$ \cite{wang_dissimilar_2024}, 
$\epsilon$ \cite{sun_neuroevolution_2023}, and amorphous configurations \cite{liu_unraveling_2023}. 
A landmark achievement was realized by Zhao et al. \cite{zhao_complex_2023}, 
who introduced a general-purpose MLIP referred to as tabGAP, capable of accurately 
describing the atomic interactions across all major \ga\ polymorphs. 
This potential has enabled large-scale simulations of 
defect evolution \cite{he_threshold_2024,he_ultrahigh_2024,liu_orientation-dependent_2025}, 
phase transformations \cite{azarov_self-assembling_2025}, 
and crystallization dynamics \cite{zhang_orientation-dependent_2023,li_edge-dependent_2025}, 
and has played a pivotal role in recent studies of irradiation effects in 
\ga\ \cite{azarov_universal_2023,zhao_crystallization_2025,han_electronic_2025,abdullaev_ions_2025}.
However, despite its considerable success, tabGAP still exhibits notable limitations compared 
to other state-of-the-art MLIPs, including relatively lower accuracy and computational 
speed, as will be demonstrated in the following sections. 
These limitations motivate us to rigorously benchmark the overall 
performance of tabGAP and develop a new, more accurate and efficient MLIP for \ga.

\section{Methodology}

\subsection{NEP architecture}

NEP, developed by Fan \textit{et al.}, consists of an ACE-like descriptor constructed using 
Chebyshev polynomials and a regression model based on 
a feedforward neural network \cite{fan_gpumd_2022,song_general-purpose_2024}. 
Its explicit atomic-environment featurization is carefully designed to strike an excellent 
balance between prediction accuracy and computational efficiency. 
Moreover, NEP is natively implemented in the Graphics Processing Units Molecular Dynamics 
(GPUMD) package, enabling an impressive computational throughput of up to $1\times10^{7}$ 
atom-steps per second on a single GPU \cite{xu_gpumd_2025}. 
Considering both its accuracy and performance, NEP stands out as one of the most promising 
MLIPs for large-scale simulations of \ga\ under a wide range of physical conditions.

\subsection{Hyperparameters}

\subsection{Datasets}

\subsection{Simulations}


\begin{acknowledgments}
We wish to acknowledge the support of the author community in using
REV\TeX{}, offering suggestions and encouragement, testing new versions,
\dots.
\end{acknowledgments}

\appendix

\section{Appendixes}

To start the appendixes, use the \verb+\appendix+ command.
This signals that all following section commands refer to appendixes
instead of regular sections. Therefore, the \verb+\appendix+ command
should be used only once---to setup the section commands to act as
appendixes. Thereafter normal section commands are used. The heading
for a section can be left empty. For example,
\begin{verbatim}
\appendix
\section{}
\end{verbatim}
will produce an appendix heading that says ``APPENDIX A'' and
\begin{verbatim}
\appendix
\section{Background}
\end{verbatim}
will produce an appendix heading that says ``APPENDIX A: BACKGROUND''
(note that the colon is set automatically).

If there is only one appendix, then the letter ``A'' should not
appear. This is suppressed by using the star version of the appendix
command (\verb+\appendix*+ in the place of \verb+\appendix+).

\section{A little more on appendixes}

Observe that this appendix was started by using
\begin{verbatim}
\section{A little more on appendixes}
\end{verbatim}

Note the equation number in an appendix:
\begin{equation}
E=mc^2.
\end{equation}

\subsection{\label{app:subsec}A subsection in an appendix}

You can use a subsection or subsubsection in an appendix. Note the
numbering: we are now in Appendix~\ref{app:subsec}.

Note the equation numbers in this appendix, produced with the
subequations environment:
\begin{subequations}
\begin{eqnarray}
E&=&mc, \label{appa}
\\
E&=&mc^2, \label{appb}
\\
E&\agt& mc^3. \label{appc}
\end{eqnarray}
\end{subequations}
They turn out to be Eqs.~(\ref{appa}), (\ref{appb}), and (\ref{appc}).

% The \nocite command causes all entries in a bibliography to be printed out
% whether or not they are actually referenced in the text. This is appropriate
% for the sample file to show the different styles of references, but authors
% most likely will not want to use it.
\nocite{*}

\bibliography{apssamp}% Produces the bibliography via BibTeX.

\end{document}
%
% ****** End of file apssamp.tex ******
